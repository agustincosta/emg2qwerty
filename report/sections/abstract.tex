\begin{abstract}
    This paper presents enhancements to the EMG2QWERTY framework for predicting keyboard typing from surface electromyography (sEMG) signals. We introduce and evaluate two architectural modifications: (1) an autoencoder-based dimensionality reduction technique that compresses 32-channel EMG spectrograms to a 16-channel representation, and (2) a multi-scale temporal depth-separable (TDS) convolutional architecture that captures dependencies at multiple time scales simultaneously. Our experiments on the EMG2QWERTY dataset demonstrate that the multi-scale TDS convolutions significantly improve performance, reducing character error rate (CER) from 20.79\% to 15.25\% with only a modest 7\% increase in parameter count. Conversely, the autoencoder-based dimensionality reduction negatively impacted performance despite achieving reasonable reconstruction error. The combination of both approaches also underperformed relative to the baseline. These findings highlight the importance of multi-scale temporal modeling for EMG-based typing prediction while suggesting that preserving the full dimensionality of EMG signals is crucial for maintaining discriminative power in this application domain.
\end{abstract}